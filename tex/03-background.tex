\documentclass[../main.tex]{subfiles}

\begin{document}
    \chapter{Background}\label{chap:background}

    This chapter provides an overview of the problems addressed in this thesis work, as well as the main technologies upon which our approach is built. \\
    Section~\ref{sec:deeplearning} is an introduction to the basics of deep learning, a branch of machine learning concerned with the study of Artificial Neural Networks (ANNs).
    This section draws heavily on~\cite{Goodfellow-et-al-2016}.

    \section{Deep Learning}\label{sec:deeplearning}

    \paragraph{Introduction}
    The hardware substrate of current digital information processors is very different from the biological substrate of the human brain.
    In fact, they are the opposite~\cite{anusuya2010superficial}.
    For this reason, many tasks that require effort and reasoning for a human to accomplish, such as chess, are pretty straightforward for machines,
    for those problems can be encoded into a set of formal rules which can then be solved algorithmically.
    Conversely, other kinds of tasks that humans do without even consciously reasoning about them are tremendously difficult for machines to accomplish.
    We find so easy to perform this kind of tasks that we don't even manage to explain how we do them. Encorporating the intuition
    and common sense that humans give for granted into machines is one of the holy grails of AI.\@ In this context, one of the main
    innovations of deep learning, and of representation learning in general, in that of letting the machine learn by itself what the
    model of the world should be, instead of manually encoding knowledge into it. At the time of this writing, this is the most promising
    direction towards artificial intelligence.

    \paragraph{History}
    At first, it was believed that intelligent behavior could be achieved by hard-coding knowledge into the system by means
    of some formal language. Reasoning would then be achieved through the use of inference rules, such as Modus Ponens.
    These were the so-called \textbf{Expert Systems}~\cite{Jackson_1986}.
    It was soon evident that the number of formal rules needed for intelligent behavior exceeds by several orders of magnitude
    the number of rules one could possibly write by hand.
    \newline
    This led to a major shift from deductive systems to inductive ones, where the machine itself \textit{learns} to extract
    knowledge from data. By seeing a lot of input-output examples in the form $y = f(x)$ it would figure out the relationship $f$ itself.
    This was the beginning of \textbf{Machine Learning} (ML). In ML, the algorithm does not see the raw data itself, but a higher level
    representation of it. Formally, a function $\phi(x)$ is provided which takes in input the raw data $x$ and outputs a representation of $x$ in the
    form of a numerical vector. This vector is then used to find the relationship between $x$ and $y = f(\phi(x); \theta)$.
    In classical ML, the input representation is hand-crafted, that is the function $\phi(x)$ is manually designed by scientists
    and engineers often over the course of decades. This was a major drawback for many reasons:
    \begin{itemize}
        \item There is little transfer between different domains, such as vision and speech. That is, a function $\phi(x)$ that is found
            to work well on one specific domain, is not able to generalize to other ones as well.
        \item The majority of efforts and resources go into the feature engineering process itself. In most cases, it is more an art
            than a science.
        \item The goodness of the features entirely depends on human instinct. Many important features may simply be ignored or not
            discovered at all by scientist.
    \end{itemize}
    The natural evolution of this is to not only let the program learn the input-output relation $y = f(x)$,
    but also the input representation as well. That is, the algorithm learns both a suitable representation of the input $\phi(x)$, and then
    the relationship between this representation and the output $y = f(\phi(x))$. This research field is called \textbf{Representation Learning}.
    \newline
    \textbf{Deep Learning} is an extension of representation learning. The machine learns multiple levels of representations
    in a hierarchical fashion, where more complex concepts are built on top of simpler ones.
    Deep Learning is nowadays behind many state-of-the-art techniques in many research fields, like computer vision and speech recognition,
    and many believes it to be one of the most promising ways to reach human level artificial intelligence someday.
    \newline
    In the following, we will provide a very concise introduction to the main concepts of deep learning. In particular, we will cover only a
    tiny subset of the whole field: \textit{feed-forward (convolutional) neural networks} for \textit{supervised learning} (classification).
	Thus, the setting is the typical one for classification problems: we have a \textit{data set} of $n$ samples, of which the i-th sample
	is $(X_{i} \in \mathbb{R}^{K}, Y_{i} \in \{1, 2, \ldots, C\})$, where $X_{i}$ is a \textit{tensor}, that is a multi-dimensional array, and $Y_{i}$
	is the class label for sample $i$. The task is to provide a model capable of predicting the correct class label $Y^{'}$ for a previously
	unseen instance $X^{'}$.

    \subsection{Feed-Forward Neural Networks}
    The feed-forward network can be thought of as a general framework in which different layers of processing units are stacked on top
    of each other. Each layer implements a function that takes in input the output of the previous layer.\footnote{Note that also the input and
    the output of the network are considered layers.} A single layer can be viewed either as implementing a vector valued function or
    as a set of parallel processing units, each of which takes in input the outputs of the previous layer units, performs some
    sort of computation, and outputs a scalar value. The name neural networks stem from the fact that these models are loosely inspired
    by how the human brain works. In this regard, the parallel processing units in each layer perform a role analogous
    to that of neurons in the brain. However, the similarity between artificial and biological neural networks is very superficial, and
    NNs should not be seen as an attempt to perfectly mimic the human brain.
    The classical example of a feed-forward network architecture is the MultiLayer-Perceptron (MLP)~\cite{mlp}, in which
    each layer implements an affine mapping of the input followed by an element-wise non-linear function:

    \begin{equation}
        h = \sigma(Wx + b)
    \end{equation}

    Where $x \in \mathbb{R}^{m \times n}$ is the layer's input, $h \in \mathbb{R}^{m \times p}$ is the output,
    $\sigma$ is the non-linearity (called the \textit{activation function}) and
    $W \in \mathbb{R}^{n \times p}, b \in \mathbb{R}^{p}$ are the layer's parameters.
    The parameters $W$ are called the \textit{weights}; in MLPs there is a different weight for each pair of input-output units.
    The intercept of the affine mapping $b$ is called the \textit{bias}, because it represents the output of the layer when
    the input is zero. Regarding $\sigma$, there exists many non-linearities in the deep learning literature, some of the most popular are the
    \textit{sigmoid or logistic} function, the \textit{hyperbolic tangent (TanH)} and the \textit{rectifier linear unit (ReLU)}:

    \begin{equation}
        Sigmoid(x) = \frac{1}{1 + e^{-x}}, TanH(x) = \frac{e^{x} - e^{-x}}{e^{x} + e^{-x}}, ReLU(x) = \max(0, x)
    \end{equation}

    The layer $h = \sigma(Wx + b)$ described above can be thought of as a way of stretching and squashing space in order to project
    the input onto a new space in which it is linearly separable, that is, there exists a hyperplane separating the different classes.
    In the following depiction we can see such a layer in action:

    \begin{figure}[h!]
        \centering
        \begin{subfigure}{\linewidth}
            \centering
        	\includegraphics[width=.5\linewidth]{img/non-linearly-separable.png}\label{fig:nonlinearsep}\hfill
        	\includegraphics[width=.5\linewidth]{img/linearly-separable.png}\label{fig:linearsep}
        \end{subfigure}
        \caption{The original data (left) is not linearly separable, that is, there does not exists a hyperplane capable of
            separating the red curve from the blue curve. The transformation $h = \sigma(Wx + b)$ project the input x into a new space
        (right) where it's easy to find a separating hyperplane. Images taken from: \url{http://colah.github.io/posts/2014-03-NN-Manifolds-Topology/}}\label{fig:linear-layer-viz}
	\end{figure}

    When we stack many such layers one on top of the other, we have a deep network. In particular, the input-output mapping defined by
    a MultiLayer-Perceptron with $n$ layers is recursively defined as:

    \begin{equation}
        \hat{y} = W_{n}(\sigma(W_{n-1}(\sigma(W_{n-2}(\ldots \sigma(W_{1} x + b_{1}) \ldots)) + b_{n-2})) + b_{n-1}) + b_{n} 
    \end{equation}

    With $W_{1}, b_{1}$ denoting the parameters of the first layer and $W_{n}, b_{n}$ the parameters of the last layer. The dimension of the
    first layers is constrained to be $(m, \cdot)$, with $m$ number of inputs, while the dimension of the last layer is constrained to be
    $(\cdot, c)$, with $c$ number of classes. The dimensions of all the other layers are hyper-parameters of the model that the designer
    of the architecture can arbitrarily chose. The number of layers $n$ is also a hyper-parameter.
    \newline
    The output vector $\hat{y}$ has one entry for each class, and it can be thought of as the scores the network assigns to different classes.
    Since this vector can assume arbitrary values, it is typically followed by a function which converts the raw class scores into a
    probability distribution over classes, that is that normalizes the scores so that they sum to one.
    The most popular function used in the deep learning literature to do that is the \textbf{softmax} function:

    \begin{equation}
        softmax(z) = \left[ \frac{e^{z_{1}}}{\sum_{i = 1}^{c} e^{z_{i}}} , \ldots , \frac{e^{z_{c}}}{\sum_{i = 1}^{c} e^{z_{i}}} \right], z = \left[ z_{1}, \ldots ,z_{c}  \right]
    \end{equation}

    Thus, the complete input-output mapping of the MLP is:

    \begin{equation} \label{eq:MLPequation}
        \hat{y} = softmax(W_{n}(\sigma(W_{n-1}(\ldots \sigma(W_{1} x + b_{1}) \ldots)) + b_{n-1}) + b_{n})
    \end{equation}

    Now that we've defined our model, we need two more things: a definition of error and a procedure to learn from such errors.

    \paragraph{The Loss Function}
    The first thing to note is that the output of the model is a vector with $c$ entries, one for each class. So we need to define the
    reference value $y$ also as a vector with $c$ entries. This is done by using the so-called \textit{one-hot-encoding}, that is, class
    $i$ is represented by a vector in which the $i$-th position is 1 and the other $c - 1$ positions are 0. For instance, in a ten class
    classification problem, class 6 would be represented as $y_{6} = \left[ 0, 0, 0, 0, 0, 1, 0, 0, 0, 0 \right] $. Having both
    the model output $\hat{y}$ and the reference value $y$, we can define the loss function. The most popular choice for classification
    tasks in deep learning is the \textbf{cross-entropy} loss function, defined as:

    \begin{equation}
        L(y, \hat{y}; \theta) = - \sum_{i = 1}^{c} \lbrack y_{i} \log(\hat{y_{i}}) + (1 - y_{i})\log(1 - \hat{y_{i}}) \rbrack
    \end{equation}

    It is easy to see that for all the entries for which $y_{i} = 0$, only the second term contributes to the loss, and for the entry
    for which $y_{i} = 1$ (the true label), only the first term contributes to the loss. It can be shown that this expression is the result
    of applying the maximum likelihood principle to the point estimation in the space of functions that is performed by neural networks. That is,
    the cross-entropy loss function represents the log-probability of classes as given by the model.
    The previous equation is the cross-entropy loss for a single input sample. In deep learning (and more generally in machine learning),
    the loss is usually averaged over a number of samples, called a \textit{mini-batch}. In this case, the loss becomes:

    \begin{equation}
        L(y, \hat{y}; \theta) = - \frac{1}{N} \sum_{i = 1}^{N} \sum_{j = 1}^{c} \left[ y_{i, j} \log(\hat{y_{i, j}}) + (1 - y_{i, j})\log(1 - \hat{y_{i, j}})  \right]
    \end{equation}

    Now we know how to compute the model's errors with respect to reference values. The last thing to cover is the design of a learning
    procedure, that is how to modify the model's parameters in order to reduce future errors.

    \paragraph{The Learning Algorithm}
    In deep learning, the most popular technique for minimizing the loss function is \textbf{gradient descent}. The basic idea
    is that the gradient of the loss function with respect to the model parameters gives us the direction of maximum increase
    of the loss. Hence, by updating the parameters in the opposite direction, we are effectively minimizing the error.
    Formally, the learning rule is the following:

    \begin{equation}\label{eq:gradientdescent}
        \theta^{k+1} = \theta^{k} - \alpha \nabla L(y, \hat{y}; \theta)
    \end{equation}

    Where $\alpha$ is a hyper-parameter of the algorithm that controls the step size in the direction of steepest descent. It is called the
    \textbf{learning rate} in the literature. This parameter is of fundamental importance in the optimization procedure, as its value often
    makes the difference between convergence and divergence of the minimization process. Equation~\eqref{eq:gradientdescent} was the most basic
    version of gradient descent. Other techniques have been developed in recent years to overcome the major limitations of this scheme.
    The momentum~\cite{momentum} method provides a sort of short-term memory of recent updates to the optimizer and it helps smoothing the
    trajectory of minimization and improving speed of convergence. More recent methods like Adam~\cite{adam} are more sophisticated, as
    they employs per-parameter adaptive learning rates.
	\newline
    Now we know how to modify the network's parameters in order to minimize the loss function, but how do we compute the gradients?
    It turns out there is a very computationally efficient algorithm which allows us to compute the partial derivatives of the loss with
    respect to the model's parameters. This algorithm is called \textbf{backpropagation}.

    \paragraph{The BackPropagation Algorithm}
    The BackPropagation algorithm~\cite{backprop} computes partial derivatives in a feed-forward neural network. Although this
	technique is mainly used in the context of neural networks optimization, the algorithm is general and can be used to differentiate
	arbitrary functions.
	\newline
	The reference value $y$ is defined with respect to the output layer of the network, so it is easy to compute the gradient of this
	layer. Hidden layers instead do not have target values, so their gradient cannot be computed directly. But we know that hidden layers
	influence the output of the last layer, because they provide the representation on top of which the output layer is built. BackPropagation
	uses this fact to recursively compute the gradients starting from the output layer and going backwards to the first hidden layer. Hence the
	name BackPropagation. In particular, BackPropagation cleverly uses the chain rule of calculus to compute the partial derivatives
	in previous layers. The chain rule of calculus basically states that if we have a nested function
    $ z = f(y) = f(g(x)) $, the derivative of z w.r.t.\ x is equal to:
    $$ \frac{dz}{dx}  = \frac{dz}{dy} \frac{dy}{dx} $$
    This is similar to the structure of equation~\eqref{eq:MLPequation}. The full learning procedure of a MultiLayer-Perceptron is showed in Algorithm~\ref{algbackpropagation}.

    \begin{algorithm}
    \caption{MultiLayer-Perceptron learning algorithm. \newline
    The \textbf{forward} function computes the output of the network given input $x$: $\hat{y} = f(x; \theta)$.
    The \textbf{backward} function computes the gradients of the loss function with respect to the model parameters. It starts with
    the gradient of the output layer, and then it propagates it backwards.
    The \textbf{update} function updates the parameters of the network according to the gradient descent update rule.
    The three functions must always be called in this exact order: forward \rightarrow{} backward \rightarrow{} update}\label{algbackpropagation}
    \begin{algorithmic}[1]

        \Require{$net$, the network}
        \Require{$(x, y)$, (input, target output)}

        \Function{forward}{$net, x, y$}
            \State{$h[0] = x$}                               \Comment{h is a tensor holding the outputs of all layers}
            \For{$k = 1, \ldots, d$}                         \Comment{d is the depth (number of layers)}
                \State{$a[k] = net.W[k] h[k-1] + net.b[k]$}
                \State{$h[k] = \sigma(a[k])$}
                \State{$net.a[k] = a[k]$}
            \EndFor{}
            \State{$\hat{y} = h[l]$}
            \State{$J(\hat{y}, y; \theta) = L(\hat{y}, y; \theta) + \lambda \Omega(\theta)$ \Comment{$\lambda \Omega(\theta)$} is the regularization term}
            \State{\textbf{return} $(\hat{y}, J(\hat{y}, y; \theta))$}
        \EndFunction{}

        \Function{backward}{$net, y$}
            \State{$g = \nabla_{\hat{y}} J(\hat{y}, y; \theta)$}     \Comment{gradient of the output layer}
            \For{$k = l, \ldots, 1$}                                 \Comment{from output to input}
                \State{$g = \nabla_{a[k]} J = g \odot f'(net.a[k])$} \Comment{gradient before the non-linearity}
                \State{$net.gradB[k] = \nabla_{net.b[k]} J = g + \lambda \nabla_{net.b[k]} \Omega(\theta)$}
                \State{$net.gradW[k] = \nabla_{net.W[k]} J = g h^{(k-1)T} + \lambda \nabla_{net.W[k]} \Omega(\theta)$}
                \State{$g = \nabla_{h^{(k-1)}} J = {net.W[k]}^{T} g$}  \Comment{propagate the gradients below}
            \EndFor{}
        \EndFunction{}

        \Function{update}{net}
            \For{$k = l, \ldots, 1$}
                \State{$net.b[k] = net.b[k] - \alpha \cdot net.gradB[k]$}
                \State{$net.W[k] = net.W[k] - \alpha \cdot net.gradW[k]$}
            \EndFor{}
        \EndFunction{}

    \end{algorithmic}
    \end{algorithm}

    \subsection{Convolutional Neural Networks}\label{subsec:convnets}
    As we saw in the previous section, in the standard feed-forward net each layer is composed by an affine mapping followed by some non-linearity.
    But this is not the only way in which layers can be designed. There exists several kinds of feed-forward architectures, and in this section
    we will see what is arguably the most popular and successful example of specialization: the Convolutional
    Neural Network (abbreviated as CNN or Convnet).
    \newline
    CNNs are a specialization of the feed-forward network, particularly suited to work with data in which spatial information is important, like
    time series in one dimension and images in two dimensions. If we have to train a standard feed-forward net to classify images, the first
    thing we would do is to transform the 2D image into a 1D vector by concatenating together all the pixel values. By doing this we are
    clearly losing the spatial information relating the pixel to one another. Convnets instead are capable of fully exploit the spatial
    correlations between pixels.
    \newline
    More formally, a Convolutional Neural Network is a feed-forward network in which at least one layer uses the convolution operation.

    \subsubsection{The Convolution Operation}
    The convolution operation employed in deep learning (also called \textit{cross correlation}) is a linear operation.
    In the case of two dimensional data, in particular images, the output at pixel $(i, j)$ is a weighted linear combination
    of a neighborhood around $(i, j)$. The set of weights is called the \textit{kernel}. It is easy to understand this operation
    by looking at picture~\ref{fig:convolution-op}:

    \begin{figure}[h!]
        \centering
    	\includegraphics[width=.8\linewidth]{img/convolution-op.png}
        \caption{The Convolution operation. Image taken from: \url{https://developer.apple.com/library/content/documentation/Performance/Conceptual/vImage/ConvolutionOperations/ConvolutionOperations.html}}\label{fig:convolution-op}
	\end{figure}

    More formally, the convolution operation for pixel $(i, j)$ is defined by the following:
    \begin{equation}
        F(i, j) = (I \ast K)(i, j) = \sum_{w} \sum_{h} I(i + w, j + h) K(w, h)
    \end{equation}
    Where $I$ is the input image, $K$ is the convolution kernel, and $F$ is the output, also called \textit{feature map}. The full
    feature map is defined by applying the previous operation for $i = \{1, \ldots, w\}$ and for $j = \{1, \ldots, h\}$, that is, the kernel
    is a window that slides over the entire image, computing the weighted combination at each pixel location. The kernel can be thought
    of as a specialized filter, which produces high responses to certain features of the input. For instance, a filter can produce a
    high response when it sees a vertical edge, or a blotch, and a low response in any other case.

    \subsubsection{Learning the kernels}
    The exists many techniques based on convolution and kernels to compute edges or other features in the input image, like the Sobel
	filter~\cite{gao2010improved} for instance.
    The main difference with those approaches is that in Conv nets, the filters are not hand-crafted, but instead the network \textbf{learns}
    the values of the filters. In particular, in each convolutional layer, several kernels are computed in parallel, with each one specializing
    in the detection of a different feature. As a consequence of the hierarchical nature of deep learning, filters in low level layers specialize
    in the detection of low-level features such as edges and blobs, while high-level filters specialize in the detection of more high-level features,
    such as faces and animals.

    \begin{figure}[h!]
    	\includegraphics[width=\linewidth]{img/learned-conv-filters.png}
        \caption{Visualizing Conv net filters, taken from~\cite{zeiler13}}\label{fig:conv-filters}
	\end{figure}

    In the next section we will see the other main building block of CNNs, namely the \textit{pooling} operation.

    \subsubsection{Pooling in Convolutional Networks}
    In CNNs, a convolutional layer is usually followed by a pooling layer, that is a way of performing dimensionality reduction
    and improving the network overall robustness to input transformation. In particular, a pooling layer replaces the output of the
    network at pixel location $(i, j)$ with a summary statistics of nearby pixels. Many statistics are usually employed in the deep learning
    literature~\cite{scherer2010evaluation}. The most popular is arguably max pooling~\cite{nagi2011max}, in which the summary statistic
	is simply the maximum value. Figure~\ref{fig:max-pooling} shows how max pooling works.

    \begin{figure}[h!]
        \centering
    	\includegraphics[width=.6\linewidth]{img/max-pooling.png}
        \caption{The max pooling operation.}\label{fig:max-pooling}
	\end{figure}

    It can be shown that the spatial pooling operation improves invariance to small translations in the input, i.e.\ if the
    input image is translated some pixels to the left or right, the output is still the same. This is a nice property to have when
    dealing with images, because we aren't usually interested in knowing precisely where a feature is located, but we are only concerned
	with whether the feature is actually present or not.

    \subsubsection{Advantages of Convolutional Networks}
    Convolutional Neural Networks have some unique features that make them particularly efficient, both from a computational viewpoint
	and from a computer vision perspective. The three main advantages of CNNs with respect to standard feed-forward networks are:
    \begin{itemize}
        \item \textbf{Sparse Interactions}: the feed-forward net is composed of affine layers $\sigma(Wx + b)$ where there is a different
            parameter for each pair of input-output units. Thus, each layer of the net has a number of parameters that is $O(m \times n)$,
            where $m$ is the number of input units and $n$ the number of output units. CNNs instead has a number of parameters that
            is $O(k \times w \times h)$, where $k$ is the number of filters and $w, h$ are respectively width and height of each filter.
            Since kernels are typically much smaller that the input, CNNs can have less parameters by orders of magnitude.

        \item \textbf{Parameter Sharing}: As we said in the previous point, feed-forward nets have a different parameter for each pair
            of input-output units. In CNNs instead, each kernel is re-used over the entire input.
            This can be viewed as a parameter sharing technique, in which we have $m$ kernels with the constraint that they have to
            share parameters. Of course, in practice, we have only one kernel that is used as a sliding window over the input,
            but thinking of it as a use of parameter sharing can give more insights into the effectiveness of CNNs.

        \item \textbf{Translation equivariance}: The convolution operation is naturally equivariant to translation, i.e.\ if the input is
            translated by a small amount of pixels to the left or right, the output is translated by the same amount.  This is clearly a
            desirable property when dealing with images. This operation is not naturally invariant to other input transformations such as 
            scaling and rotation, but CNNs as a whole can achieve such properties with appropriate mechanisms, albeit CNNs invariance
			is valid only for small transformations. In fact, making CNNs invariant to larger transformations is one of the main motivations
			behind this work.
    \end{itemize}

\end{document}
